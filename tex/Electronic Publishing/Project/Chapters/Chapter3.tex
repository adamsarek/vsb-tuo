\chapter{Implementace}
\label{chap:4}
\section{Vykreslování grafického výstupu}
\label{sec:4.1}
\subsection{Zavedení Canvasu}
\begin{sloppypar*}
Pro vykreslování grafického výstupu je využíván HTML element \texttt{<canvas>}.
V JavaScriptu je objekt tohoto elementu možné získat pomocí DOM (objektový model dokumentu). \cite{web:MDN/DOM}
Získaný objekt \texttt{HTMLCanvasElement} je použit v rámci vlastní vytvořené třídy \texttt{WhiteboardCanvas}, která byla vytvořena za účelem optimálního vykreslování všech artefaktů tabule.
Nad objektem \texttt{HTMLCanvasElement} je zavolána funkce \texttt{getContext("2d")}, která vrací instanci rozhraní \texttt{CanvasRenderingContext2D}.
% CanvasRenderingContext2D => dvourozměrného vykreslovacího kontextu
Tato instance je pak dále nastavena všem již přidaným objektům tabule, které díky ní mohou začít vykreslovat.
\end{sloppypar*}

%\section{Využité zabudované funkce pro vykreslování}
%\begin{description}
%	\item[CanvasRenderingContext2D.clearRect()] -- funkce je využita k mazání obsahu tabule
%	\item[CanvasRenderingContext2D.fillText()] -- funkce je využita k vykreslování textu poznámky
%	\item[CanvasRenderingContext2D.fillStyle()] -- funkce je využita ke stylizaci výplně kreslených objektů tabule
%\end{description}



\subsection{Vykreslovací cyklus}
\begin{sloppypar*}
Vykreslovací cyklus probíhá v rámci již zmíněné třídy \texttt{WhiteboardCanvas}.
Cyklus začíná v metodě \texttt{linkHtml(canvas)} po získání instance \texttt{CanvasRenderingContext2D} zavoláním funkce \texttt{redraw()}.
\end{sloppypar*}
\begin{lstlisting}[language=JavaScript,label=src:JavaScript/WhiteboardCanvas.redraw(),caption={Metoda redraw() třídy WhiteboardCanvas}]
redraw() {
	if(this.redrawRequested) {
		this.clear();
		this.draw();
		this.redrawRequested = false;
	}

	requestAnimationFrame(this.redraw.bind(this));
}
\end{lstlisting}
\begin{sloppypar*}
Funkce začíná podmínkou, která kontroluje, zda je požadováno překreslení aktuálního stavu plátna.
Požadavek je při spuštění nastaven na hodnotu \texttt{true} právě z důvodu, aby počáteční vykreslení proběhlo bez nutnosti požadavek dále nastavovat.
\end{sloppypar*}
\begin{sloppypar*}
Pokud je podmínka splněna, dojde k vyčištění plátna a následnému vykreslení objektů přidaných funkcí \texttt{setObjects(objects)}.
Dále je pak požadavek nastaven na \texttt{false}, jelikož po aktuálním překreslení již další nejsou potřeba, dokud nedojde k nějaké změně, v rámci které bude tento požadavek vyvolán.
\end{sloppypar*}
\begin{sloppypar*}
Ať už podmínka byla nebo nebyla splněna, po jejím průchodu je, díky \texttt{requestAnimationFrame()}, funkce \texttt{redraw()} pokaždé znovu volána ve frekvenci odpovídající obnovovací frekvenci monitoru uživatele.
Díky omezení frekvence nemůže dojít k vyšším nárokům stránky (např. v módu kreslení), než které je koncové zařízení schopno zvládnout.
\end{sloppypar*}


\subsubsection{Odeslání požadavku o vykreslení}
\begin{sloppypar*}
Pokud dojde ke změně kterékoliv vlastnosti jakéhokoliv objektu tabule nebo se změní její barevný motiv je potřeba vyvolat požadavek o překreslení plátna skrze \texttt{requestRedraw()}.
\end{sloppypar*}
\begin{lstlisting}[language=JavaScript,label=src:JavaScript/WhiteboardCanvas.requestRedraw(),caption={Metoda requestRedraw() třídy WhiteboardCanvas}]
requestRedraw() {
	this.redrawRequested = true;
}
\end{lstlisting}
\begin{sloppypar*}
Nastavením proměnné \texttt{redrawRequested} na hodnotu \texttt{true} dochází v rámci vykreslovacího cyklu ke splnění podmínky, ve které následně dojde k překreslení plátna tabule.
\end{sloppypar*}



\subsection{Využití více vrstev pláten}
V rámci optimalizace byly místo jednoho Canvasu vytvořeny dva, jeden pro mřížku na pozadí a druhý pro samotný obsah tabule.
Důvod je jednoduchý. Plátno s mřížkou se aktualizuje méně často než hlavní vlákno.
Pokud by tedy byla mřížka s hlavním obsahem v jednom canvasu, musela by se pokaždé znovu překreslit i tato mřížka a zbytečně by tak byl snížen výkon webu.
%, což je redundantní proces, který by zbytečně snižoval výkon.
Plátno s vlastním obsahem tabule má průhledné pozadí, aby byla mřížka na pozadí viditelná.



\subsection{Využití vláken}
\begin{sloppypar*}
Součástí optimalizace řešení je využití současné práce více vláken.
JavaScript umožňuje kromě hlavního vlákna, které slouží především pro vykreslování a event handling, využít také další vlákna pomocí web workerů.
Tyto workery nemají přístup ke všem objektům, ke kterým má přístup hlavní vlákno, takže je nutné určité potřebné data poslat jiným způsobem.
To však není problém, jelikož workery mají přímo zabudovaný event \texttt{onmessage} pro příjem zpráv a funkci \texttt{send()} pro odesílání zpráv.
\end{sloppypar*}
Vedlejší vlákna jsou vhodná především pro větší výpočty či práci s daty, které nutně nemusí být v hlavním vláknu.
Čím méně instrukcí musí hlavní vlákno řešit, tím lépe zvládá činnosti, které je povinné zajišťovat.
Výsledkem je tedy plynulejší vykreslování a rychlejší reakce na jednotlivé události.

V této práci je využíváno pouze jedno vedlejší vlákno, jelikož je pro optimální funkčnost dostačující, nicméně je možné, že s pomocí více vedlejších vláken by šlo určité operace provádět současně a tedy efektivněji.


\subsubsection{Výpočet souřadnic}
Nejvíce výpočtů v práci souvisí s výpočtem souřadnic a je tedy vhodné tyto výpočty provádět v rámci vedlejšího vlákna.
Souřadnice jsou důležitou součástí zaznamenávání a zobrazování objektů tabule.
Práce k výpočtu souřadnic využívá několik faktorů, které ovlivňují, kde se zrovna uživatel nachází, jaká část tabule je viditelná a zda je vůbec samotný výpočet potřeba.

Výchozím referenčním bodem tabule je bod [0;0].
Tento bod označuje místo, kde se objevují noví uživatelé při prvním spuštění tabule a nachází se uprostřed uživatelovy obrazovky, ať už na počítači či mobilním zařízení.

Výhodou umístění bodu na střed je to, že je možné uprostřed vytvořit obsah, který má být pro nové uživatele na první pohled viditelný bez nutnosti se přepínat mezi záchytnými body tabule či se po tabuli přesunovat ručně.
Je to také místo, kde bude chtít začít většina uživatelů pracovat, takže bude výsledek viditelný ihned jednoduše pro všechny.

Dalším důvodem výběru je také to, že pokud by byl výchozí bod sice uprostřed tabule, ale tabule by souřadnice počítala pouze v kladných hodnotách, tak by objekty okolo středu měly zbytečně celkově vyšší hodnoty souřadnic.
Může se to zdát jako zbytečnost, nicméně při vyšším počtu tabulí, uživatelů a jejich dat toto může být důležitým opatřením pro udržení nižší datové náročnosti jak pro provozující server, tak pro internetové připojení klienta.
K tomuto přispívá nejen bod [0;0], ale také fakt, že veškeré souřadnice tabule jsou zaznamenávány v celých číslech.

Dalším rozhodnutím pro nižší datový přenos je omezení maximální velikosti tabule.
%Pro zajištění nižší náročnosti nejen pro přenos dat, ale také pro relativně rychlý export tabule je omezení maximální velikosti tabule.
Stávající maximální velikost tabule byla stanovena na rozlišení 32K z toho důvodu, že se dnes ještě nejedná o běžně dosažitelné rozlišení a tedy není snadné jednoduše vyčerpat celý prostor tabule.
Zároveň se však jedná o stále relativně nízké rozlišení pro exportování tabule do obrázku.
%Výhodou výběru tohoto bodu právě na střed je to, že je možné libovolně rozšiřovat obsah tabule do všech okolních směrů.
%Vybrat výchozí bod na střed je výhodné, jelikož se jedná o místo, kde lze vytvářet obsah viditelný pro nové uživatele a je možné se na tento bod kdykoliv přesunout pomocí tlačítka výběru důležitých míst tabule.
%, který je zároveň také místem, kde se objevují noví uživatelé, je bod [0;0].

\subsubsubsection{Faktor pozice uživatele}
Aby bylo možné vykreslit mřížku či libovolný objekt tabule, je nutné znát aktuální pozici uživatele.
Při prvním spuštění je tato pozice v bodě [0;0] nicméně díky možnosti se po tabuli pohybovat se tato hodnota v průběhu používání mění.
Pozice uživatele je vždy dána souřadnicemi bodu tabule uprostřed vykreslovaného plátna.
Při každé změně pozice je nutné souřadnice všech objektů přepočítat a znovu vykreslit aktualizovaný obsah.

\subsubsubsection{Faktor přiblížení}
Nejen pozice uživatele, ale i všech artefaktů tabule je ovlivněna také mírou přiblížení.
Přiblížit je možné až na úroveň 300~\% a oddálit pak na 30~\% standardní míry zobrazení.
S každým přiblížením či oddálením je rovněž nutné znovu přepočítat a vykreslit změny.

\subsubsubsection{Faktor velikosti obrazovky}
Velikost Canvasu ovlivňuje souřadnice událostí pracujících s pozicí kurzoru či prstu na obrazovce.
Vždy je nutné od aktuální pozice ukazatele odečíst polovinu velikosti Canvasu, čímž dojde k vycentrování ukazatele na střed a k dalšímu využití momentálních souřadnic uživatele.

Canvas také svojí velikostí udává viditelnou oblast tabule, kterou je nutné vykreslit, čímž je možné spoustu objektů tabule vynechat a nutné výpočty provádět pouze u zobrazených objektů.
Aby byl objekt považován za viditelný, je nutné aby byl alespoň jeden roh objektu ve viditelné oblasti.
Možným vylepšením současného stavu projektu je doplnění této podmínky o kontrolu skutečné vykreslené oblasti objektu.
Aktuálně se totiž může stát, že pokud např. objekt volného kreslení v daném rohu neobsahuje žádnou čáru, nicméně tento roh je součástí viditelné oblasti, tak se s tímto objektem počítá jako s viditelným a je nutné u něj vypočítat souřadnice i když ve výsledku na tabuli není vidět.
%zda se objeví objekt


\subsubsection{OffscreenCanvas}
OffscreenCanvas je experimentální technologie, která nově přináší vykreslování Canvasu pomocí Web workeru.
Kontext plátna není přímo závislý na DOM, což umožňuje jeho použití k vykreslování mimo hlavní vlákno. \cite{web:MDN/OffscreenCanvas}
%Bez použití této technologie renderování zmrazí veškeré ostatní operace v main threadu, dokud neproběhne vykreslování.
Přenesením renderovacích příkazů na vedlejší vlákno se eliminuje sekání způsobené vyčkáváním hlavního vlákna na dokončení vykreslovacích operací, které při plném využití výkonu znemožňují další interakci s webem.

Tato technologie je s použitím 2D kontextu zatím mezi nejrozšířenějšími prohlížeči podporovaná pouze prohlížeči postavenými na projektu Chromium a dále prohlížečem Samsung Internet, které globálně používá okolo 70\% uživatelů. \cite{web:CanIUse/OffscreenCanvas}

Jelikož podpora nezahrnuje prohlížeč Mozilla Firefox a další široce zastoupené značky, tak bylo po dohodě s vedoucím práce rozhodnuto tuto technologii neimplementovat a pouze její existenci zmínit v rámci tohoto dokumentu.

Podpora technologie OffscreenCanvas je však již nyní mezi vývojáři velmi žádaná a bude důležitou součástí webů využívajících Canvas.



\subsection{Vlastní třídy pro vykreslování}
Standardní metody rozhraní CanvasRenderingContext2D jsou plně postačující pro vykreslení několika jednoduchých tvarů.
Avšak pro využití ve větším počtu je v rámci zachování efektivity příkazů a přehlednosti zdrojového kódu lepší tyto funkce sjednotit.
Ke sjednocení byly použity třídy, které JavaScript podporuje od verze standardu ECMAScript 6.
%Nejedná se o jediný způsob, avšak tento byl vybrán z toho důvodu, že tyto třídy podporují dědičnost. Díky dědičnosti lze k problémům přistupovat pomocí OOP, což lépe strukturalizuje a zjednodušuje výsledný kód.

Seznam níže uvádí třídy, které byly v bakalářské práci vytvořeny za účelem zjednodušení či zefektivnění vykreslování jednotlivých artefaktů tabule:
\begin{sloppypar*}
\begin{description}
	\item[WhiteboardCanvasPolyline] -- Třída slouží k vykreslování lomené čáry. Tento typ čáry je využit k vykreslení mřížky a také v rámci nástroje kreslení, kde jedna lomená čára vyobrazuje jeden souvislý tah.\\
	V metodě \texttt{draw()} třída shlukuje příkazy \texttt{CanvasRenderingContext2D.moveTo()} a \texttt{CanvasRenderingContext2D.lineTo()} v rámci cyklu, který prochází zadané pole počátečních a koncových souřadnic čar. Cyklus čáry přidává do jedné cesty uvedené voláním \texttt{CanvasRenderingContext2D.beginPath()}, což zajišťuje efektivnější vykreslování, než kdyby byla každá čára vykreslována samostatně. \cite{web:HTML5Rocks/CanvasImprovePerformance}
	\item[WhiteboardCanvasRoundedRectangle] -- Třída slouží k vykreslování zaobleného obdélníku, který je použit jako pozadí poznámky. Výběr zaobleného obdélníku místo hranatého je čistě stylizační rozhodnutí.\\
	Třída v metodě \texttt{draw()} obsahuje, po volání \texttt{CanvasRenderingContext2D.beginPath()}, sekvenci 4 příkazů \texttt{CanvasRenderingContext2D.arc()}, které tvoří kruhové oblouky využité na zaoblené hrany obdélníku. \cite{web:MDN/CanvasRenderingContext2D/arc} Oblouky jsou následně spojeny zavoláním funkce \texttt{CanvasRenderingContext2D.closePath()}. \cite{web:MDN/CanvasRenderingContext2D/closePath} Spojení oblouků pomocí příkazu \texttt{closePath()} je efektivnější oproti samostatnému vykreslování zbylých čar na každé straně obdélniku. Celá nově vzniklá oblast je, dle nastavení v \texttt{CanvasRenderingContext2D.fillStyle}, vyplněna metodou \texttt{CanvasRenderingContext2D.fill()}. \cite{web:MDN/CanvasRenderingContext2D/fill}
	\item[WhiteboardCanvasText] -- Třída slouží k vykreslování textu poznámky.\\
	Canvas neumožňuje vykreslovat text na více řádků, takže je zdrojový HTML kód zpracován a rozdělen na jednotlivé řádky. Zarovnání dále nabízí pouze od daného bodu, takže je nutné pro zarovnání doprava a na střed tento výchozí bod posunout doprava resp. na střed poznámky. Jednotlivé řádky textu jsou poté zobrazeny pomocí metody \texttt{CanvasRenderingContext2D.fillText()}.
	\item[WhiteboardCanvasNote] -- Třída slouží k vykreslování poznámky. Obsahuje objekty tříd \texttt{WhiteboardCanvasRoundedRectangle} a \texttt{WhiteboardCanvasText}, díky kterým je možné poznámku vykreslit a obsahuje metody, jako např. \texttt{setPosition(x, y)} nebo \texttt{setSize(w, h)}, které oběma objektům nastaví potřebné vlastnosti. Cílem bylo zapouzdřit jednotlivé části poznámky do jedné třídy, pro snadnější úpravu libovolných parametrů.
	\item[ImageWrapper] -- Třída slouží k indikaci stavu načtení obrázku a umožňuje nastavit vlastní události \texttt{onerror(fn)} a \texttt{onload(fn)}. Dále obsahuje metodu \texttt{isReady()}, která říká, zda je obrázek načten (platí i při chybě při načítání) a \texttt{canDraw()}, která vrací \texttt{true} pouze v případě, pokud byl obrázek načten v pořádku.
	\item[WhiteboardCanvasImage] -- Třída slouží k vykreslování obrázku. Využívá objekt třídy \texttt{ImageWrapper}, jehož metody \texttt{isReady()} a \texttt{canDraw()} určí aktuální stav obrázku, který je pak následně zobrazen pomocí \texttt{CanvasRenderingContext2D.drawImage()}. Obrázek může nabývat následujících stavů:
	\begin{itemize}
		\item \textbf{Obrázek se načítá} -- je zobrazeno tmavě šedé pozadí s neutrálním piktogramem obrázku
		\item \textbf{Obrázek se nepodařilo načíst} -- je zobrazeno tmavě červené pozadí s piktogramem rozbitého obrázku
		\item \textbf{Obrázek byl úspěšně načten} -- je zobrazen samotný obrázek
	\end{itemize}
	\item[WhiteboardCanvasShape] -- Třída slouží k vykreslování geometrického tvaru.
	\begin{table}[h!]
		\centering
		\caption[Výpis tvarů a jejich volaných funkcí CanvasRenderingContext2D]{Výpis tvarů a jejich volaných funkcí CanvasRenderingContext2D}
		\label{tab:WhiteboardCanvasShape/CanvasRenderingContext2D/Functions}
		\begin{tabular}{ll}
			\toprule
			Tvar & Volané funkce CanvasRenderingContext2D\\
			\midrule
			Obdélník & \begin{sloppypar*}\texttt{beginPath()}, \texttt{rect()}, \texttt{fill()}\end{sloppypar*}\\
			Elipsa & \begin{sloppypar*}\texttt{beginPath()}, \texttt{ellipse()}, \texttt{fill()}\end{sloppypar*}\\
			Pravoúhlý trojúhelník & \begin{sloppypar*}\texttt{beginPath()}, \texttt{moveTo()}, \texttt{lineTo()}, \texttt{closePath()}, \texttt{fill()}\end{sloppypar*}\\
			Rovnoramenný trojúhelník & \begin{sloppypar*}\texttt{beginPath()}, \texttt{moveTo()}, \texttt{lineTo()}, \texttt{closePath()}, \texttt{fill()}\end{sloppypar*}\\
			Kosočtverec & \begin{sloppypar*}\texttt{beginPath()}, \texttt{moveTo()}, \texttt{lineTo()}, \texttt{closePath()}, \texttt{fill()}\end{sloppypar*}\\
			\bottomrule
		\end{tabular}
	\end{table}\\
	Tvary mají společné vlastnosti jako pozici, rozměry, pozadí a liší se pouze samotným typem tvaru.
	Jejich vykreslovací funkce tedy mohou být jednoduše rozděleny na základě typu tvaru přímo v rámci jedné třídy.
	Tímto se eliminuje duplikování totožných metod při vytváření samostatných tříd pro každý tvar zvlášť.
\end{description}
\end{sloppypar*}



\subsection{Barevné motivy tabule}
Dnešním trendem moderních webů a aplikací je podpora tmavého motivu.
Zatímco světlý motiv je vhodné využít především za ostřejšího denního světla, tak tmavý se hodí spíše v šeru večerních či nočních hodin, kdy je pro oči šetrnější.

Právě s denní dobou počítá ve výchozím stavu také vypracované řešení, kdy vždy v 18:00 se dle uživatelova lokálního času přepne web do tmavého motivu a ráno během 6:00 se zase opět vrátí do světlého motivu.
Toto automatické přepínání jde samozřejmě kdykoli tlačítkem přepnout přímo na výběr konkrétního motivu či vrátit zpátky na automatickou volbu.


\subsubsection{Barevný kontrast}
Vzhledem k tomu, že tmavé objekty na tmavém motivu a stejně tak světlé objekty na světlém motivu jsou špatně viditelné, tak se nabízí několik variant, jak tento problém vyřešit.

Jedním ze způsobů je zvolení daného motivu při vytváření tabule s tím, že by byl daný motiv nastaven pro všechny uživatele tabule a bez možnosti jej dále přepínat.
Tento způsob však není příliš ideální, jelikož si tímto uživatel de facto vyřadí právě jednu z funkcí, které web nabízí.

Dále je poté možné implementovat pouze světlý (nebo případně pouze tmavý) motiv.
Tento přístup lze pozorovat u téměř všech webů, kterými se práce inspirovala.
V rámci řešení však byla snaha najít vhodnou variantu, jak by se dalo implementovat oba barevné motivy a zároveň zachovat určitý barevný kontrast samotných objektů.

Při implementaci řešení tedy došlo k rozhodnutí, že aby byl obsah plátna dobře viditelný ať už na tmavém či na světlém motivu, které bude možné libovolně přepínat, tak bude potřeba zavést určité barevné odstíny.

\subsubsubsection{Zpracování barevného kontrastu}
\begin{figure}[h!]
	\centering
	\includegraphics[width=0.5\textwidth]{Figures/colorMode.pdf}
	\caption{Příklad zpracování barvy \#4de6b4}
	\label{fig:colorMode}
\end{figure}

\begin{sloppypar*}
Jednotlivé objekty v rámci jejich vytváření či editace nabízí možnost změny barvy.
Pro výběr barvy je využit HTML element \texttt{<input type="color">}, který barvu vrací v hexadecimálním formátu \#rrggbb o barevné hloubce 24 bitů.
\end{sloppypar*}
Pro zesvětlení či ztlumení dané barvy je vrácený řetězec nejdříve převeden do 12 bitové barevného hloubky.
Tento krok je proveden převedením jednotlivých segmentů R, G a B do desítkové soustavy, vydělením 51, aritmetickým zaokrouhlením, vynásobením 3 a následně převedením zpět do hexadecimální soustavy.
%Jednotlivé části barev R (červená), G (zelená) a B (modrá) nabývají hodnot 00-FF, což po jejich převedení do desítkové soustavy znamená 0-255.
%Toto převedené číslo je následně vyděleno 51, aritmeticky zaokrouhleno, vynásobeno 3 a dále převedeno zpátky do hexadecimální soustavy.
Výsledkem těchto operací je hexadecimální řetězec, jehož 1.--2., 3.--4. a 5.--6. pozice obsahují stejný znak, a proto je možné jej zkrátit do formátu \#rgb.

Zkrácená varianta umožňuje vykreslit pouze 4~096 rozdílných barev, což je oproti původním 16~777~216 barvám veliký rozdíl.
Pokud by se jednalo o nástroj pro profesionální práci s grafikou, tak by tato barevná ztráta byla absolutně nepřípustná.
V tomto řešení však umožňuje právě již zmíněné zesvětlení či ztlumení barvy tím, že do zkrácené verze řetězce se opět, do každé ze 3 barev, přidá vždy zleva jeden znak.
Přidávané znaky zachovávají poměr jednotlivých barev od 0 do 5 a jsou případně zvýšené o požadovanou hodnotu od 0 do 10 tak, aby byly dostatečně viditelné v rámci daného motivu.
Obrázek \ref{fig:colorMode} graficky popisuje celý proces zpracování barev na uvedeném příkladu.

\subsubsubsection{Kontrast jednotlivých typů objektů}
Barevný odstín textu poznámky je stejný jako u vykreslovaných čar volného kreslení.
To stejné platí i o pozadí poznámky a pozadí geometrického tvaru.

Barevný odstín textu poznámky se pohybuje ve stejném rozmezí jako vykreslované lomené čáry u volného kreslení.
Důvodem stejného rozmezí je především charakter daných artefaktů, jelikož oba jsou velmi úzké a proto na tmavém pozadí by měly být světlejší a zároveň na světlém pozadí tmavší.

Pozadí poznámky i geometrického tvaru má rovněž stejné rozmezí, což je dáno tím, že zabírají relativně větší plochu a navíc u poznámky je nutné vzít v potaz také kontrast nejen vůči plátnu, ale také vůči samotnému textu poznámky.

%\subsubsection{Odstíny jednotlivých nástrojů}
%Tyto odstíny jsou součástí téměř všech typů objektů tabule kromě obrázku a při změně motivu se změny projeví přebarvením daného objektu do odpovídajícího odstínu dle barvy, kterou objekt obsahuje.



\subsection{Indikace upravovaných objektů}
Aktuálně upravované objekty tabule lze jednoduše rozeznat díky tomu, že jsou během úpravy barevně ohraničeny.
Barva ohraničení se liší na základě několika faktorů.
Pokud se jedná o objekt upravovaný aktuálním uživatelem v aktuálním okně, je ohraničení sytě růžové.
V případě, že jej sice upravuje aktuální uživatel avšak v jiném okně, je ohraničení zbarveno do šedě růžové barvy.
Šedé je pak ohraničení v případě, že daný objekt edituje úplně jiný uživatel.
Vykreslení barevného rámečku je čistě kosmetická záležitost, avšak samotný výběr objektu uživatelem je důležitá součást integrity tabule, která je dále podrobněji vysvětlena v následující kapitole.

\begin{figure}[h!]
	\centering
	\includegraphics[width=1\textwidth]{Figures/ObjectIndication1.png}
	\caption{Pohled uživatele X v okně A (poloviční mřížka, světlý motiv, Mozilla Firefox)}
	\label{fig:ObjectIndication1}
\end{figure}

\begin{figure}[h!]
	\centering
	\includegraphics[width=1\textwidth]{Figures/ObjectIndication2.png}
	\caption{Pohled uživatele X v okně B (plná mřížka, tmavý motiv, Microsoft Edge)}
	\label{fig:ObjectIndication2}
\end{figure}

\begin{figure}[h!]
	\centering
	\includegraphics[width=1\textwidth]{Figures/ObjectIndication3.png}
	\caption{Pohled uživatele Y v okně C (žádná mřížka, světlý motiv, Mozilla Firefox)}
	\label{fig:ObjectIndication3}
\end{figure}
\clearpage




\section{Datová komunikace se serverem}
\label{sec:4.2}
Důležitou součástí webu je možnost spolupráce více uživatelů zároveň v reálném čase.
Sdílení dat je umožněno WebSocket serverem, který přijímá aktualizace jednotlivých detailů tabule a dále je rozesílá všem aktivně připojeným uživatelům kromě uživatele, který tuto změnu inicioval.
Součástí práce serveru je také ukládání dat do databáze či do proměnných o čemž pojednává následující sekce \ref{sec:4.3}.

%\section{Propojení jednotlivých částí webu}



\subsection{Součásti WebSocket serveru}
\subsubsection{Node.js}
Pro vývoj serverové části práce bylo vybráno prostředí Node.js.
Jedná se o asynchronní runtime JavaScriptu, který právě díky asynchronnímu přístupu umožňuje nenáročný vývoj škálovatelných síťových aplikací. \cite{web:NodeJS/About}
Node.js aplikace se programují v jazyce JavaScript a tyto aplikace je možné díky aktivní Node.js komunitě obohatit o již vytvořené balíčky, které celý vývoj zjednodušují.
Práce využívá ke svému fungování 4 balíčky uvedené níže:
\begin{itemize}
	%\item \textbf{express} -- slouží ke spuštění HTTP serveru na určitém síťovém portu, na který se poté klient připojuje
	\item \textbf{google-auth-library} -- určen k ověření klientského identifikačního čísla získaného po přihlášení k uživatelskému účtu Googlu
	\item \textbf{http} -- slouží ke spuštění HTTP serveru na určitém síťovém portu, na který se poté klient připojuje
	\item \textbf{mysql} -- umožňuje navázat komunikaci s MySQL databází
	\item \textbf{ws} -- zajišťuje komunikaci s klientským zařízením pomocí protokolu WebSocket
\end{itemize}


\subsubsection{WebSocket}
Technologie WebSocket byla vybrána z důvodu dlouhodobého spolehlivého spojení skrze plně duplexní komunikační kanál. \cite{book:AndrewLombardi/WebSocket}
Protokol vychází ze standardu RFC 6455 z roku 2011 a jeho aktuální podpora se pohybuje okolo 97--98~\% všech uživatelů webových prohlížečů. \cite{web:IETF/rfc6455}\cite{web:CanIUse/WebSockets}
%WebSocket protokol poskytuje plně duplexní komunikační kanál skrze jedno HTTP spojení a jeho definice vychází ze standardu RFC 6455 z roku 2011. \cite{book:AndrewLombardi/WebSocket}
%WebSocket API umožňuje jednoduché využití WebSocket protokolu pro komunikaci uživatelova prohlížeče a serveru. \cite{web:MDN/WebSocketAPI}


\subsubsection{JSON}
\begin{sloppypar*}
WebSocket protokol umožňuje přenos binárních či textových dat (kódování UTF-8). \cite{web:IETF/rfc6455}
K výměně dat práce používá datového formátu JSON, který pro přenos převádí na text pomocí příkazu \texttt{JSON.stringify()}.
JSON byl zvolen z důvodu snazšího debugování a vyšší přehlednosti výsledného kódu.
\end{sloppypar*}
Nižší velikosti odesílaných a přijímaných dat by bylo možné dosáhnout využitím binárních dat, nicméně jejich využití je vhodné po ujasnění výsledné formy všech posílaných dat.
JSON umožňuje rychleji a jednodušeji provádět změny struktury posílaných dat.



\subsection{Připojení uživatele}
Po vytvoření a spuštění je server připraven navázat spojení a komunikaci s webovou stránkou, pokud tato stránka splní určité přístupové kritéria.


\subsubsection{Validace údajů}
\subsubsubsection{Lokální část}
\begin{sloppypar*}
Prvním krokem před samotným pokusem o spojení je validace přístupových dat.
Nejprve je provedena kontrola URL adresy, tedy konkrétně, zda je uživatel připojen přes zabezpečený HTTPS protokol a zda celé doménové jméno odpovídá očekávaní.
Dále je pak ověřen řetězec přístupového režimu, který v adrese nabývá hodnot \textit{user}, \textit{edit} či \textit{view} a v případě posledních dvou také desetimístný unikátní identifikátor tabule pro daný režim.
Identifikátor se skládá z velkých a malých písmen anglické abecedy, číslic od 0 do 9 a dále pomlčky a podtržítka a je ověřen pomocí regexu \texttt{/\textasciicircum[A-Za-z0-9\textbackslash-\textbackslash\_]\{10\}\$/}.
Jako poslední je také provedena kontrola, zda se uživatel přihlásil a zda po tomto přihlášení byl získán potřebný token k dalšímu ověření na serveru.
\end{sloppypar*}
Toto základní ověření dat zajišťuje ochranu serveru před nežádoucími pokusy o připojení, především v případě možné interní chyby aplikace.
Pokud byly všechny výše uvedené podmínky splněny, webová stránka zažádá o připojení na WebSocket server a do připojované URL adresy zahrne GET požadavky s uvedenými kontrolními daty.

\subsubsubsection{Serverová část}
Na WebSocket server se může připojit prakticky kdokoliv, kdo zná jeho adresu, která je na webu lehce dohledatelná ve vývojářských nástrojích.
Nikdy tedy není možné předpokládat, že se uživatel připojuje pouze z konkrétního webu a také, že posílá smysluplné data.
Je tedy vhodné všechny vstupy ověřovat a při detekci jakékoliv anomálie uživatele odpojit od serveru buď úplně či případně s možností se znovu automaticky připojit v případě nějaké chyby aplikace.
\begin{sloppypar*}
Na serveru je tedy nejprve potřeba zkontrolovat původ připojovaného klienta, tedy opět zda se připojuje skrze HTTPS, z očekávaného webu, s použitelným režimem přístupu a v případě identifikátoru také jeho formátová správnost.
Token vrácený po přihlášení uživatele je následně pomocí balíčku \texttt{google-auth-library} verifikován a pokud je tato akce úspěšná, tak je získán unikátní identifikátor uživatelského účtu Google tzv. \texttt{sub}.
Po verifikaci je \texttt{sub} uložen do databáze (pouze v případě nového uživatele).
Poslední kontrola se týká pouze přístupových režimů edit a view a zjišťuje, zda se unikátní identifikátor tabule nachází v databázi.
%, ze které je pomocí příkazu \texttt{SELECT user\_id FROM user WHERE sub = \textquotesingle\$\{sub\}\textquotesingle LIMIT 1} vráceno uživatelské ID
%Je tedy nutné mít k přijímaným datům určitou míru nedůvěry a ověřovat co možná nejvíce  a je tedy nezbytné
%Pokud některá z předešlých podmínek není splněna, tak web dále nepokračuje v pokusu o připojení na WebSocket server.
%Toto základní ověření dat zajišťuje vyšší bezpečnost tím, že server chrání před nežádoucími pokusy o připojení.
%Jedná se však spíše o ošetření možných problémů aplikace než o pokročilou obranu před hackery.
%Práci je možné rozšířit mimo jiné právě o implementaci vyšší úrovně zabezpečení nejen na lokální úrovni.
\end{sloppypar*}


\subsubsection{Režimy přístupu}
\subsubsubsection{Režim uživatelského profilu}
- Přidání/editace/smazání více tabulí

\subsubsubsection{Režim editace tabule}
- Editace parametrů tabule (pouze její autor)\\
- Výběr objektu - editace, pohyb, smazání

\subsubsubsection{Režim zobrazení tabule}
- Osekaná verze režimu editace



\subsection{Udržování spojení}
- Ping/Pong
%\section{Příklady odesílaných dat}



\subsection{Důvody a řešení odpojení uživatele}

\subsubsection{Odpojení na základě chybné komunikace}

\subsubsection{Odpojení spojené s problémovým připojením k serveru}
- Výpadek serveru\\
- Výpadek internetového připojení uživatele
\clearpage

% výchozí pozice ve středu tabule je výhodná, jelikož:
% - umožňuje v budoucnu navyšovat maximální velikost tabule, která se zvětší do všech směrů a zároveň objekty i uživatelé zůstanou na původním místě (nebude třeba přepočítávat ze starých menších tabulí)
% - je možné se dále pohybovat do všech směrů
% - lze využít pro 
% - objekty u středu mají souřadnice blížící se k 0;0 a jsou tedy nižší než kdyby bylo možné se pohybovat pouze jedním směrem, kde by někteří uživatelé museli daleko zajíždět od počátku, aby mohli kreslit (ušetření dat)
% objekty jsou následně přepočítány do souřadnic canvasu


%Hlavním cílem je především zaměření na optimalizaci a efektivitu příkazů pro nerušené využití při spolupráci více uživatelů.
%Data jsou průběžně ukládána a načítána z databáze a pro jejich komunikaci mezi zařízeními je využito protokolu WebSocket.

% Další nápady:
% - kopírování objektů
% - přesouvání objektů do popředí po označení
% - místa uživatelů + přímo vytvořené checkpointy
% - změna velikost objektu
% - označení více objektů v rámci vybrané oblasti
% - přidání možnosti registrace pomocí emailu místo googlu a úprava profilu
% - přidání více možností přihlášení - microsoft, facebook, apple, github
% - ukládání do jiných formátů než jenom do JSON - např. XML
% - sdílení kromě emailu také přes facebook, twitter, instagram a další
% - přidání dalších jazyků kromě češtiny a angličtiny
% - přidání ikony/obrázku tabule, který by byl v pozvánce, ikoně záložky a na dalších místech
% - přidání chatu
% - přidání poznámek
% - offscreenCanvas

% Dotazy:
% - datum u citace odkazu - datum změny na webu nebo datum kdy jsem použil?
% - mám dobře dané citace v textu?
% - můžu takhle mít obrázky?




\section{Správa dat}
\label{sec:4.3}
\subsection{Správa dat na serveru}
\subsubsection{MySQL databáze}
% Pozice uživatele na tabuli


\subsubsection{Dočasné proměnné serveru}



\subsection{Lokální správa dat}
\subsubsection{Export tabule v nativní podobě formátu JSON}


\subsubsection{Nahrání obrázku pomocí Fetch API a PHP}


\subsubsection{Uložení uživatelských nastavení}
\subsubsubsection{LocalStorage}

\subsubsubsection{PHP Session}
\clearpage




\section{Uživatelské prostředí}
\label{sec:4.4}
\subsection{Uživatelské rozhraní}
\subsubsection{Řešení responzivity}


\subsubsection{Barevné motivy GUI}


\subsubsection{Jazyky GUI}



\subsection{Jednotlivé stránky webu}
\subsubsection{Stránka uživatelského účtu}
- Umožňuje přidání více tabulí...\\
- U každé tabule má tlačítka s odkazy na režim editace a zobrazení a smazání...


\subsubsection{Stránka s tabulí}
- Tlačítko pro přesouvání na důležité místa\\
- Tlačítka pro export a sdílení\\
- Tlačítko pro přepnutí na uživatelskou stránku či odhlášení\\
- Ovládání přiblížení tabule - zoom in/out/reset, fullscreen\\
- Tlačítka uživatelských nastavení - mřížka, barevné motivy, jazyk\\
- Tlačítka nástrojů tabule - kreslení, poznámka, obrázek, geometr. tvar, nativní podoba\\
- Rozdíly uživatelského rozhraní režimu editace a zobrazení


\subsubsection{Chybová stránka}
- Zobrazuje chyby 403, 404 a další...



\subsection{Sdílení tabule}
- Velmi krátká kapitola pouze o sdílení tabule, které nesedí do jiných kapitol\\
- Možnost zkopírování odkazu (bez dalších kroků, nejedná se o adresní řádek, ale o výběr možnosti v <select>)


\subsubsection{Sdílení emailem}
- Sdílení předem vytvořené zprávy v nastaveném jazyku - automaticky pomocí serveru (PHP funkce mail())\\
- Sdílení předem vytvořené zprávy v nastaveném jazyku - ručně uživatelem (mailto:)
\endinput